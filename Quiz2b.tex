\documentclass{article}

\usepackage[utf8]{inputenc}

\usepackage{nicefrac}
\usepackage{amssymb, amsmath, amsfonts}
\usepackage{amsthm}
\usepackage{tikz}
\usetikzlibrary{matrix,shapes,arrows}
\usepackage{pgfplots}
\usepgfplotslibrary{groupplots}
\usepackage[a4paper, margin=1in]{geometry}

\newtheorem{proposition}{Proposition}
\newtheorem{theorem}{Theorem}
\newtheorem{definition}{Definition}
\newtheorem{lemma}{Lemma}
\newtheorem{conjecture}{Conjecture}
\newtheorem{corollary}{Corollary}
\newtheorem{remark}{Remark}
\newtheorem{assumption}{Assumption}

\newlength\figureheight
\newlength\figurewidth
\setlength\figureheight{8cm}
\setlength\figurewidth{14cm}

\newcommand{\tikzdir}[1]{tikz/#1.tikz}
\newcommand{\inputtikz}[1]{\input{\tikzdir{#1}}}

\DeclareMathOperator*{\argmin}{arg\; min}     % argmin
\DeclareMathOperator*{\argmax}{arg\; max}     % argmax
\DeclareMathOperator*{\tr}{tr}     % trace
\DeclareMathOperator{\Cov}{Cov}
\DeclareMathOperator{\logdet}{log\;det}

\title{EE3011 Modeling and Control\\Quiz 4}
\date{}
\begin{document} \maketitle

\begin{enumerate}
\item The Bode magnitude plot of a minimum phase system is shown in Figure~\ref{fig:1}. Determine an approximate mathematical model in the form of transfer function for the system.
  \begin{figure}[ht]
\setlength\figureheight{8cm}
\setlength\figurewidth{14cm}
    \centering
    \inputtikz{Quiz23}
    \caption{Magnitude Plot\label{fig:1}}
  \end{figure}

  {\bf Solution:}
\newpage
\item Consider a feedback system with the following open-loop transfer function:
  \begin{align*}
    G(s) = K\frac{s-1}{s^2+2s+1}.
  \end{align*}
  The Bode plots of $G(s)$ when $K = 1$ are shown in Figure~\ref{fig:2}. Sketch the Nyquist plot. Using Nyquist Stability Criterion, find the range of $K$ that makes the system closed-loop stable.
  \begin{figure}[ht]
\setlength\figureheight{5cm}
\setlength\figurewidth{14cm}
    \centering
    \inputtikz{Quiz24}
    \caption{Bode Plot of $(s-1)/(s^2+2s+1)$\label{fig:2}}
  \end{figure}

  {\bf Solution:}
\end{enumerate}

\end{document}
%%% Local Variables:
%%% TeX-command-default: "Latexmk"
%%% End:
